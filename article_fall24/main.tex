\documentclass{./springer/svjour3}
\usepackage{graphicx} % This lets you include figures
\graphicspath{ {./figures/} }
\usepackage[rightcaption]{sidecap}
% \usepackage{subcaption}
\usepackage{wrapfig}
\usepackage{float}
\usepackage{imakeidx}
\usepackage{comment}
\usepackage{commath}
\usepackage[titletoc]{appendix}
\usepackage{graphicx}
\usepackage{resizegather}
\usepackage[english]{babel}
% \usepackage{subcaption}
\usepackage{tabu}
\usepackage{booktabs}
\usepackage{xfrac}
\usepackage{tabularx}
% \usepackage{amssymb}
\usepackage{amsmath}
% \usepackage{amsthm}
\usepackage{commath}
\usepackage{graphicx,bm}
\usepackage{verbatim}
% \usepackage{caption}
\usepackage{lscape}
\usepackage{relsize}
\usepackage{enumitem}
\usepackage{textcomp}
\usepackage{breqn}
\usepackage{makecell}
\usepackage{longtable,tabularx}
\usepackage{multirow}
\usepackage{doi}
\usepackage{fancyhdr}
\usepackage{algorithm}
\usepackage{algpseudocode}
\usepackage{setspace}
\usepackage{footnote}
\PassOptionsToPackage{hyphens}{url}
\usepackage{hyperref}
\hypersetup{colorlinks,linkcolor={blue},citecolor={blue},urlcolor={blue}}
\usepackage[numbers]{natbib}
\usepackage{mathtools}
%\usepackage[disable]{todonotes}
\usepackage[framemethod=tikz]{mdframed}
\usepackage{booktabs,xcolor,siunitx}
\usepackage{soul}
% \usepackage{cleveref}
\usepackage[small, compact]{titlesec}
\usepackage{xcolor}
\usepackage{appendix}
% \usepackage{geometry}
% \geometry{
% a4paper,
% total={170mm,257mm},
% left=20mm,
% top=20mm,
% }

% \newtheorem{theorem}{Theorem}[section]
% \newtheorem{lemma}[theorem]{Lemma}
\newcommand{\ra}[1]{\renewcommand{\arraystretch}{#1}}
\newcommand{\degree}{\ensuremath{^\circ\,}}
\newcommand{\overbar}[1]{\mkern 1.5mu\overline{\mkern-1.5mu#1\mkern-1.5mu}\mkern 1.5mu}
\newcommand{\f}[2]{\frac{#1}{#2}}
\newcommand{\mb}[1]{\mathbf{#1}}
\newcommand{\tr}[1]{\mathrm{Tr}\left({#1}\right)}
\newcommand{\mbg}[1]{\boldsymbol{\mathbf{#1}}}
\DeclarePairedDelimiter{\ceil}{\lceil}{\rceil}
\DeclareMathAlphabet\mathbfcal{OMS}{cmsy}{b}{n}
\renewcommand{\d}{\mathop{}\!\mathrm{d}} % total derivative
\newcommand{\p}{\partial}

%\usepackage[section]{placeins}


\title{Summary of Fall 24 Research}
\author{Max Howell}
\institute{University Of Tennessee Knoxville$^*$ ($^*$corresponding author),
          \email{mhowel30@vols.utk.edu} \\
        \\
          \at MABE, University of Tennessee, Knoxville,
          \at Nathan W. Dougherty Engineering Building, 1512 Middle Dr, Knoxville, TN 37916\\
}

\date{\today}


\begin{document}
\maketitle{}
\bibliographystyle{ieeetr}
\begin{abstract}

fdas

\end{abstract}

\section{Introduction}

Control Co-Design (CCD) is an engineering design strategy used to optimize the design of a system with control in mind. CCD can be acheived using various control schemes, 
including open loop control [1], PID control, and other optimal control methods such as LQR or MPC Control. CCD has a very wide range of applications such as in 
aerospace systems [2] and in thermal energy storage [3]. While CCD has been used in various industries there has yet to be a compreherensive tool developed
to preform robust CCD using differentiable optimal control strategies such as nonlinear MPC control.

This research focuses on developing a tool which can be used in order to preform robust CCD using MPC control. This research is primarly focused on robotic applications and uses 
the MuJoCo [4] physics simulation engine in order to simulate the robotic systems. In this paper, do-mpc [5], a library for nonlinear MPC control, is used to simplify the 
implementation of MPC control. This research builds off of previous work [6] which used the Dymos Optimal Control library [7] in order to preform CCD of a simple 
two legged walking robot known as the compass gait. 

This portion of research can be broken down into three phases; phase one involved using the work done in [6] and implementing MPC control on both the two legged walker 
as well as a slightly more complicated three legged walker. The purpose of this phase was to understand how to implement MPC control on a simplified model of a robotic system.
Phase two of this research consisted of using the knowledge of MPC control from phase one and implementing it in a more complex system using the MuJoCo physics engine.
Phase three consists of using the MPC controlled MuJoCo system and preforming CCD on the system as well as implementing more complex algorithms used in robotics such as 
path planning and obstacle avoidance. Phase three of this research is incomplete and is only discussed as a future work in this paper.

\section{Methods}

\subsection{Phase I: Compass Gait and Three Legged Walker}

Phase I consisted of building on the research done in [6] and implementing MPC control on the compass gait system as well as a slightly more complex three legged walker.
The compass gait is a simple two legged walking robot which is able to passively walk down an incline with no control input. The three legged walker is a similar system 
but is more realistic as it uses a knee joint to walk. The compass gait as well as three legged walker are hybrid systems, meaning they change dynamics based on the state of the system.
The compass gait is shown in Figure \ref{fig:compass_gait}; for a more detailed look at the dynamics of the compass gait, see [6].

\begin{figure}[!h]
  \centering
  \includegraphics[width=8cm]{./figures/compassgaitmodel.png}
  \caption{Schematic of the compass gait model. Angles $\theta_{sw}$ and $\theta_{st}$ are shown as the swing leg angle and stance leg angle, respectively.}
  \label{fig:compass_gait}
\end{figure}

The three legged walker system is shown in Figure \ref{fig:threeleg}. For a more detailed look at the dynamics of the three legged walker, see [8]. In both the three legged walker and
compass gait, there is a single control input applied to the hip.

In order to implement MPC control on the compass gait, do-mpc was used with the optimization problem as shown in \ref{eq:twolegopt} where the states of the system ($\mb{x}$)
are shown in \ref{eq:twolegstates}. In the objective function $x_N$ and $x_k$ represent the terminal and the current state of the system, respectively.
$x_r$ represents the target state of the system, which is determined to be the state at which both the swing and stance legs are in contact with the ground. This state 
is shown in Table \ref{tab:twolegterm}.

\begin{equation}
  \begin{aligned}
  &\text{Objective: $arg min_{x_k,u_k} (x_N - x_r)^TQ_N(x_N - x_r) + \sum_{k = 0}^{N-1} (x_k - x_r)^TQ(x_k - x_r) + u_k^TRu_k $}\\
  &\text{Subject to:}\\
  &\text{Dynamic Equations of System}\\
  &\text{State Constraints: $-90^{\circ} < x_1 < 90^{\circ }, -90^{\circ} < x_2 < 90^{\circ }  $}
  \end{aligned}
  \label{eq:twolegopt}
\end{equation}

\begin{equation}
  \mb{x} = 
  \begin{bmatrix}
  \\x_{sw}\\
  \\x_{st}\\
  \\\dot{x}_{sw}\\
  \\\dot{x}_{st}\\
  \end{bmatrix}, \quad
  \label{eq:twolegstates}
\end{equation}

\begin{table}[h]
  \centering
  \caption{Target states during MPC control of the compass gait.}
  \begin{tabular}{lr}
  \toprule
  State & Target Value \\
  \midrule
  $x_{sw}$ & 0.19 \\
  $x_{st}$ & -0.3 \\
  \end{tabular}
  \label{tab:twolegterm}
\end{table}


\begin{figure}[!h]
  \centering
  \includegraphics[width=8cm]{./figures/threeleg.png}
  \caption{Diagram of the three legged walker model where q1, q2, and q3 are the stance, swing, and knee joint angles.}
  \label{fig:threeleg}
\end{figure}

The $Q_N$ and $Q$ matrices are the weighting matrices for the MPC control and are simply set to be equal to the identity matrix in this case. 
The $R$ matrix is the weighting matrix for the control input and is set to be equal to 10 to penalize the single control input.

MPC control was implemented on the three legged walker in a similar manner to the compass gait, except there were two phases considered for this system, a two link and a 
three link phase. A diagram showing these phases is shown in Figure \ref{fig:threelegstates}. These phases were treated as seperate systems with different governing equations and 
each phase was controlled using a seperate MPC controller. The transition between the two phases was triggered by "knee-strike", where the 
knee joint angle became equal to the angle of the swing leg, and "heel-strike", where the swing leg comes into contact with the ground. The 
transition between the phases were calculated using the transition equations from [8].
\begin{figure}[!h]
  \centering
  \includegraphics[width=8cm]{./figures/threelegstates.png}
  \caption{Diagram showing the two link and three link phases of the three legged walker.}
  \label{fig:threelegstates}
\end{figure}

The optimization problem for the three link phase is shown in \ref{eq:threelinkphaseopt} where the states of the system ($x$) consist of 
the stance, swing, and knee joint angles as well as the angular velocities of these joints. The optimization problem for the the two link phase is 
shown in \ref{eq:twolinkphaseopt} where the states of the system ($x$) consist of the stance and swing leg angles as well as the angular velocities of these joints.

\begin{equation}
  \begin{aligned}
  &\text{Objective: $arg min_{x_k,u_k} (x_N - x_r)^TQ_N(x_N - x_r) + \sum_{k = 0}^{N-1} (x_k - x_r)^TQ(x_k - x_r) + u_k^TRu_k $}\\
  &\text{Subject to:}\\
  &\text{Control Range: $-3 < tau < 3$}\\
  &\text{Dynamic Equations of Three Link Phase}\\
  &\text{State Constraints: $x_2 > x_3 $}
  \end{aligned}
  \label{eq:threelinkphaseopt}
\end{equation}

\begin{equation}
  \begin{aligned}
  &\text{Objective: $arg min_{x_k,u_k} (x_N - x_r)^TQ_N(x_N - x_r) + \sum_{k = 0}^{N-1} (x_k - x_r)^TQ(x_k - x_r) + u_k^TRu_k $}\\
  &\text{Subject to:}\\
  &\text{Control Range: $-3 < tau < 3$}\\
  &\text{Dynamic Equations of Two Link Phase}\\
  &\text{State Constraints: $x_2 = x_3, \dot{x_2} = \dot{x_3}$}
  \end{aligned}
  \label{eq:twolinkphaseopt}
\end{equation}

Similarly to the compass gait control problem, the $Q_N$ and $Q$ matrices are the identity matrix and the $R$ matrix is set to 1 in this case.
The target states for each phase are the states which were pre-determined to be the knee strike and heel strike states and are shown in Table \ref{tab:threelegterm}.

\begin{table}[h]
  \centering
  \caption{Target states during MPC control of the three legged walker.}
  \begin{tabular}{lrr}
  \toprule
  State & Knee Strike Target & Heel Strike Target\\
  \midrule
  $x_{sw}$ & -.106  & -0.29\\
  $x_{st}$ & 0.326 & 0.19\\
  $x_{knee}$ & 0.326 & -0.19\\
  \end{tabular}
  \label{tab:threelegterm}
\end{table}

\subsection{Phase II: MuJoCo MPC}
Phase II of this research consisted of using the knowledge of MPC which was gained from phase I and applying it to a more complex system in the MuJoCo physics engine which can 
simulate complicated robotic systems such as the humanoid shown in \ref{fig:humanoid}. Instead of applying MPC control to the humanoid, MPC control was first applied to a simpler system 
in a way which can be easily scaled up to the humanoid. The chosen system was the classic cart pole control problem, where a pole with a 
mass attatched to the end is attatched to a cart which can move back and forth in one direction. This is shown in \ref{fig:cartpole}. The goal of this problem is to 
control the pole to balance upright on the cart when the cart is at a stationary set location. This is done by applying a force to the cart to move it in the positive or 
negative x direction.

\begin{figure}[!h]
  \centering
  \includegraphics[width=8cm]{./figures/humanoid.png}
  \caption{Example of a humanoid robot which can be simulated in the MuJoCo physics engine.}
  \label{fig:humanoid} 
\end{figure}

\begin{figure}[!h]
  \centering
  \includegraphics[width=8cm]{./figures/cartpole-dynamics.png}
  \caption{Diagram of the cart pole system.}
  \label{fig:cartpole} 
\end{figure}

The cart pole system has previously been solved with various control techniques using the governing equations of motion for the system. The equations are derived in [9]. In this
paper, the cart pole problem was solved using both MPC control with the equations of motion and using MPC control in MuJoCo in order to compare the results.
The states for each cart pole problem are shown in \ref{eq:cartpolestates} and the initial and target states of this problem are shown in
\ref{tab:init_tar_carteqn}.

\begin{equation}
  \mb{x} = 
  \begin{bmatrix}
  \\x\\
  \\\theta\\
  \\\dot{x}\\
  \\\dot{\theta}\\
  \end{bmatrix}, \quad
  \label{eq:cartpolestates}
\end{equation}

\begin{table}[h]
  \centering
  \caption{Initial and Target States of the Cart Pole Problem.}
  \begin{tabular}{lrr}
  \toprule
  State & Initial & Target\\
  \midrule
  $x$ & 0  & 0\\
  $\theta$ & 0 & $\pi$ \\
  $\dot{x}$ & 0 & 0\\
  $\dot{\theta}$ & 0 & 0\\
  \end{tabular}
  \label{tab:init_tar_carteqn}
\end{table}

Both methods of solving the cart pole problem used a similar optimization problem as shown in \ref{eq:cartpoleopt}. 
Solving the cart pole problem using the equations of motion was straight forward as the equations of motion were well known and could be easily implemented in do-mpc.
However, solving the cart pole problem in MuJoCo required a different approach in order to obtain the equations of motion for the states as the equations of motion 
were assumed to be unknown as they would be in the case of a humanoid robot or other complex system.

\begin{equation}
  \begin{aligned}
  &\text{Objective: $arg min_{x_k,u_k} (x_N - x_r)^TQ_N(x_N - x_r) + \sum_{k = 0}^{N-1} (x_k - x_r)^TQ(x_k - x_r) + u_k^TRu_k $}\\
  &\text{Subject to:}\\
  &\text{Control Range: $-35 < F < 35$}\\
  &\text{Equations of motion for respective method}\\
  &\text{State Constraints: $-3.5 < x < 3.5$}
  \end{aligned}
  \label{eq:cartpoleopt}
\end{equation}

In order to obtain the equations of motion for the MuJoCo cart pole system, a finite difference approach was used in order to solve for the linearized equations of motion at 
each time step in the simulation. These equations took the form of \ref{eq:mj_eqn_form}, where the A and B matrices were equal to the jacobian of the system 
with respect to the states and 
control, respectively.

\begin{equation}
  \dot{x} = Ax + Bu
  \label{eq:mj_eqn_form}
\end{equation}

The A and B matrices were found using the finite difference method by first obtaining the nominal (current, found from MuJoCo simulation)
state of the system as denoted in \ref{eq:nomstate}
and then perturbing each state one at a time as shown in \ref{eq:pertA} (repeat for all columns of the square matrix A) where $\epsilon$
is a small number in order to obtain the A matrix. The B matrix was 
obtained in a similar manner by perturbing the control input as shown in \ref{eq:pertB}.

\begin{equation}
  x_{nom} = f(x^0, \theta^0, \dot{x}^0, \dot{\theta}^0, F^0)
  \label{eq:nomstate}
\end{equation}

\begin{equation}
  A \text{ (Column One)} =\frac{ f(x^0 + \epsilon, \theta^0, \dot{x}^0, \dot{\theta}^0, F^0) - f(x^0, \theta^0, \dot{x}^0, \dot{\theta}^0, F^0)}{\epsilon}
  \label{eq:pertA}
\end{equation}

\begin{equation}
  B =\frac{ f(x^0, \theta^0, \dot{x}^0, \dot{\theta}^0, F^0 + \epsilon) - f(x^0, \theta^0, \dot{x}^0, \dot{\theta}^0, F^0)}{\epsilon}
  \label{eq:pertB}
\end{equation}

After determining the A and B matrices for the cart pole MuJoCo system, the problem could then be solved by integrating the MuJoCo engine with the do-mpc interface.
However, since the A and B matrices were being calculated at each time step, the dompc problem needed to be re-initialzed at each time step with the updated A and B matrices.
This caused computational inefficency and will be addressed in future research.

\section{Results}

\subsection{Phase I: Compass Gait and Three Legged Walker}


\end{document}
